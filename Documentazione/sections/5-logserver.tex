\documentclass[../DocumentazioneProgetto.tex]{subfiles}

\graphicspath{{./resources/capitolo1/}{../../AAA_Common_Resources/}}

\begin{document}
	%Il server log 
	\section{Il server log} 
	\label{sec:Rsyslog}
	Lo scopo di un server di log è quello di raccogliere log da altre macchine e raggrupparli in un unico posto.
%Dockerfile 
\subsection{Dockerfile} 
\label{sec:RsyslogDockerfile}
Il server log rsyslog verrà implementato senza usare un immagine docker pre-compilata, installando le componenti manualmente.
\begin{lstlisting}[language=XML, caption=Dockerfile Rsyslog] 
FROM ubuntu:21.10

RUN echo -e "\t\Updating system and installing rsyslog" \
&& apt-get update \
&& apt-get install --no-install-recommends -y rsyslog \
&& apt-get clean \
&& rm -rf /var/lib/apt/lists/*

RUN echo -e "\t\tCopying Config"
COPY Contents/rsyslog.conf /etc/rsyslog.conf

ENTRYPOINT ["rsyslogd", "-n"]\end{lstlisting}
%Analisi Dockerfile 
\subsubsection{Analisi Dockerfile} 
Partiamo caricando un immagine di \textit{ubuntu:21.10} da \textit{Docker Hub}, su questa immagine,
dopo aver aggiornato le sorgenti, installiamo il server \textit{rsyslog}.\\
Una volta installato il server facciamo pulizia del garbage creato dall'installazione, carichiamo il config di \textit{rsyslog} e
impostiamo come punto di partenza il comando \textit{rsyslogd -n}.  


%Servizio docker compose 
\subsection{Servizio docker compose} 
\begin{lstlisting}[caption=Rsyslog Docker Compose] 
Syslogserver:
	build: Dockerfiles/rsyslog/.
	image: syslogserver
	container_name: Syslog
	volumes:
		- "[PERCORSO COMPLETO CARTELLA LOG LOCALE]:/var/log"
	ports:
		- 514:514
		- 514:514/udp
	cap_add:
		- SYSLOG\end{lstlisting}
La prima riga indica il nome univoco del servizio.\\
Riga 2 è opzionale e indica il percorso in cui effettuare la build dell'immagine, se questa non è presente.\\
Riga 3 indica il nome dell'immagine. Se non è presente in locale verrà o presa dalla repo remota o buildata (se è presente l'istruzione build).\\
Riga 4 indica un nickname per il servizio.\\
Riga 6 mappa una directory locale in cui salvare i log alla directory remota \textit{/var/log}. Su questa cartella locale saranno salvati i log ricevuti dalle macchine\\
Riga 8 e 9 Aprono la port 514 in TCP e UDP per consentire al server di ricevere i log.\\
Se si intende usare solo uno dei protocolli (TCP o UDP), la porta relativa all'altro protocollo va eliminata.\\
Riga 11 specifica che il server ha bisogno di permessi aggiuntivi di tipo \textit{SYSLOG}, per info su questi permessi consultare \textit{man 7 capabilities}. 
		
%Configurazione 
\subsection{Configurazione} 
\begin{lstlisting}[language=Bash, caption=File di configurazione Rsyslog] 
# Commentare per disabilitare UDP logging
#module(load="imudp")
#input(type="imudp" port="514")

# Commentare per disabilitare TCP logging
module(load="imtcp")
input(type="imtcp" port="514")

# Template nome file log remoto
template(name="RemoteDirTemplate" type="string" string="/var/log/remote/%$year%/%$Month%/%$Day%/%$Hour%-%APP-NAME%.log")
# Regole di log
if ($source != "localhost") then {
	action(type="omfile" dynaFile="RemoteDirTemplate")
}\end{lstlisting}
In questa configurazione abilitiamo solo la versione TCP del servizio di log, per abilitare anche UDP è necessario rimuovere il commento dalle righe 2 e 3.\\
Alla riga 3 e 7 definiamo le porte per il servizio di log rispettivamente UDP e TCP, queste porte possono esssere modificate ma DEVONO corrispondere a quelle definite alle righe 8 e 9 nella \autoref{sec:RsyslogDockerfile}.\\
La riga 10 definisce il template per il nome dei file su cui salvare i log remoti, verrà analizzata a parte nella \autoref{sec:RsyslogTemplate}.\\
Le righe 12 e 13 applicano il template definito alla riga 10 solo ai log provenienti da sorgenti esterne, ovvero con l'attributo \textit{source} diverso da \textit{localhost}.  

%Template nome file 
\subsubsection{Template nome file} 
\label{sec:RsyslogTemplate}
Il template per il nome di file è il seguente: \\
\textit{/var/log/remote/\%\$year\%/\%\$Month\%/\%\$Day\%/\%\$Hour\%-\%APP-NAME\%.log}

Possiamo suddividere il template in 3 parti:
\begin{enumerate}
	\item \textit{/var/log/remote/}
	\begin{itemize}
		\item Percorso FISSO della cartella root su cui salvare i log.
	\end{itemize}
	\item \textit{\%\$year\%/\%\$Month\%/\%\$Day\%/}
	\begin{itemize}
		\item Percorso VARIABILE della cartella finale su cui salvare i log.
		\item Dipende da:
		\begin{itemize}
			\item \textit{\$year}
			\item \textit{\$Month}
			\item \textit{\$Day}
		\end{itemize}
	\end{itemize}
	\item \textit{\%\$Hour\%-\%APP-NAME\%.log}
	\begin{itemize}
		\item Nome del file in cui salvare i log
		\item Dipende da:
		\item \begin{itemize}
			\item \textit{\$Hour}
			\item \textit{\$APP-NAME}
			\begin{itemize}
				\item Identificativo del programma remoto da cui sono originati i log
				\item Può essere sostituito con \textit{\$fromhost}, l'hostname della sorgente (o indirizzo ip se DNS non disponibile).
			\end{itemize}
		\end{itemize}
	\end{itemize}
\end{enumerate}
Se, ad esempio, la macchina con il programma \textit{pippo} generasse un log il 01/01/1970 alle ore 00:05, il percorso finale verrebbe ad essere:\\
\textit{\textit{/var/log/remote/1970/01/01-pippo.log}}\\
È stato scelto questo ordine delle variabili arbitrariamente, raccogliere i log per data e ora e, in seguito per macchina, consente di avere una migliore visione di insieme.\\
Altre alternative valide sarebbero potute essere:
\begin{itemize}
	\item \textit{/var/log/remote/\%APP-NAME\%-\%\$year\%/\%\$Month\%/\%\$Day\%/\%\$Hour\%.log}
	\begin{itemize}
		\item Suddivide prima per macchina e, successivamente, per data.
		\item Fornisce una migliore visione temporale per le singole macchine ma peggiore visione di insieme sul sistema completo.
	\end{itemize}
	\item \textit{/var/log/remote/\%\$year\%/\%\$Month\%/\%\$Day\%/\%\$Hour\%.log}
	\begin{itemize}
		\item Ignora l'attributo \textit{APP-NAME}, raccoglie i log di tutte le macchine nello stesso file, suddivisi per data.
		\item Visione d'insieme sul sistema completo MA rischio di generare file molto pesanti e di difficile lettura. 
	\end{itemize}
	\item Qualunque altra configurazione con le variabili presenti sopra e altre dalla \href{https://www.rsyslog.com/doc/master/configuration/properties.html}{documentazione ufficiale rsynclog}
\end{itemize}
%Ricerca di un file di log 
\subsection{Ricerca di un file di log} 
Usando il template definito sopra, per cercare un file di log si può usare il seguente script bash:
\begin{lstlisting}[language=Bash, caption=Script per ricercare log dati specifici parametri] 
#!/bin/sh

HOST="WS1"
YEAR=""
MONTH=""
DAY=""

LIMIT="5" # Numero massimo di elementi da visualizzare
SEPARATOR="/" # / su sistemi base Unix o Darwin, \ su sistemi base MS-DOS
BASE_DIR="./remote" # Directory di partenza

if [ -z "$HOST" ]; then
	HOST=".*"
fi

if [ -z "$YEAR" ]; then
	YEAR="[0-9][0-9][0-9][0-9]"
fi

if [ -z "$MONTH" ]; then
	MONTH="[0-9][0-9]"
fi

if [ -z "$DAY" ]; then
	DAY="[0-9][0-9]"
fi

if [ -z "$BASE_DIR" ]; then
	BASE_DIR="."
fi

REGEX=".*$SEPARATOR$YEAR$SEPARATOR$MONTH$SEPARATOR$DAY$SEPARATOR[0-9][0-9]-$HOST.log"

if [ -z "$LIMIT" ]; then
	find $BASE_DIR -regex $REGEX
else
	find $BASE_DIR -regex $REGEX | head -$LIMIT
fi\end{lstlisting}
\end{document}  