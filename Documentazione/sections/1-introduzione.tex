\documentclass[../DocumentazioneProgetto.tex]{subfiles}

\graphicspath{{./resources/capitolo1/}{../../AAA_Common_Resources/}}

\begin{document}
	\section{Introduzione al progetto}
	L'obiettivo del progetto è creare una serie di webserver con accesso regolato tramite load balancer e un server di log centralizzato, il tutto sfruttando il sistema di containerizzazione \textbf{Docker}.
	%Webserver 
	\subsection{Webserver} 
	\begin{aquote}{F. Romano - \textit{web.pdf}}
		Un webserver è un'applicazione software che, in esecuzione su un (host) server, è in grado di gestire le richieste di trasferimento di pagine web verso un client, di solito un web browser.
	\end{aquote}
	%Webserver Distribuito 
	\subsubsection{Webserver Distribuito} 
	Content Here
	%Load Balancer 
	\subsection{Load Balancer}
	\begin{aquote}{Wilipedia - \textit{Bilanciamento del carico}}
	Il load balancing è una tecnica utilizzata nell'ambito dei sistemi informatici che consiste nel distribuire il carico di elaborazione di uno specifico servizio tra più server, aumentando in questo modo scalabilità e affidabilità dell'architettura nel suo complesso.
	\end{aquote}
	Un \textbf{load balancer} è un software atto ad implementare una tecnica di load balancing. 
	%Docker 
	\subsection{Docker} 
	\begin{aquote}{F. Romano - \textit{docker.pdf}}
		Docker è un progetto opensource nato con lo scopo di automatizzare e semplificare la distribuzione di applicazioni.
	\end{aquote}
	Docker si basa sul concetto di \textit{Container} 
	%Docker compose 
	\subsubsection{Docker compose} 
	Content Here
	%Log Centralizzato 
	\subsection{Log Centralizzato} 
	Content Here
\end{document}  