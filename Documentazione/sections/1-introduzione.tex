\documentclass[../DocumentazioneProgetto.tex]{subfiles}

\graphicspath{{./resources/capitolo1/}{../../AAA_Common_Resources/}}

\begin{document}
	\section{Introduzione al progetto}
	L'obiettivo del progetto è creare una serie di webserver con accesso regolato tramite load balancer e un server di log centralizzato, il tutto sfruttando il sistema di containerizzazione \textbf{Docker}.
	%Repo Github 
	\subsection{Repo Github} 
	L'intero progetto si può trovare su Github all'indirizzo \href{https://github.com/ale-ben/LAS2021-NGINX-LoadBalancer}{https://github.com/ale-ben/LAS2021-NGINX-LoadBalancer}
	%Struttura del sistema 
	\subsection{Struttura del sistema} 
	\label{sec:IntroduzioneStrutturaSistema}
	Il sistema è composto da 3 componenti:
	\begin{itemize}
		\item Webserver
		\begin{itemize}
			\item Elabora e risponde a richieste dei browser
		\end{itemize}
		\item Load Balancer
		\begin{itemize}
			\item Reindirizza le richieste dei browser ai webserver
		\end{itemize}
		\item Log Server
		\begin{itemize}
			\item Raccoglie i log di tutti i servizi in un unico posto
		\end{itemize}
	\end{itemize}
	%Struttura documentazione 
	\subsection{Struttura documentazione}
	Nella \autoref{sec:ProgettoCompleto} verrà presentato il progetto completo e verranno analizzati i file nell'insieme.\\
	Dalla  \autoref{sec:Webserver} alla \autoref{sec:Rsyslog} verranno analizzati i vari servizi nel dettaglio, con le relative configurazioni e immagini docker.\\
	Nella \autoref{sec:DockerSwarm} verrà fatto un approfondimento su un sistema alternativo di load balancing: Docker Swarm
\end{document}  