\documentclass[../DocumentazioneProgetto.tex]{subfiles}

\graphicspath{{./resources/capitolo1/}{../../AAA_Common_Resources/}}

\begin{document}
	%Il Load Balancer NGINX 
	\section{Il Load Balancer NGINX} 
	A questo servizio si collegheranno tutti i client che necessitano di accedere alla risorsa.
	%Dockerfile 
	\subsection{Dockerfile} 
\begin{lstlisting}[language=XML, caption=Dockerfile Load Balancer NGINX] 
	FROM nginx:1.20.0
	RUN echo -e "\t\tCopying Config"
	COPY Contents/nginx.conf /etc/nginx/nginx.conf
	RUN echo -e "\t\Setting Permissions"
	RUN chmod 555 /etc/nginx/nginx.conf\end{lstlisting}
	%Analisi Dockerfile 
	\subsubsection{Analisi Dockerfile} 
	Il file è uguale a quello visto nella \autoref{sec:WebserverDockerfile} tranne che per il fatto che non carichiamo i file del webserver.
	%Servizio docker compose 
	\subsection{Servizio docker compose}
	\label{sec:LoadBalancerDockerCompose}
	\begin{lstlisting}[caption=Load Balancer Docker Compose] 
LoadBalancer:
	build: Dockerfiles/load-balancer/.
	image: loadbalancer
	container_name: LB
	networks:
		- Internal
		- External
	ports:
		- "80:80"\end{lstlisting}
		La prima riga indica il nome univoco del servizio.\\
		Riga 2 è opzionale e indica il percorso in cui effettuare la build dell'immagine se questa non è presente.\\
		Riga 3 indica il nome dell'immagine. Se non è presente in locale verrà o presa dalla repo remota o buildata (se è presente l'istruzione build).\\
		Riga 4 indica un nickname per il servizio.\\
		Riga 6 impone al container di collegarsi al network \textit{Internal}, il quale non ha accesso alla rete esterna.\\
		Riga 7 Consente al container di collegari al network \textit{External}, il quale ha accesso alla rete esterna.\\
		Riga 9 Specifica il mapping delle porte in ingresso, la porta del container deve coincidere con quella specificata nel config del Load Balancer.
	\paragraph{Per implementare un sistema multi host} 
	La riga 6 non è necessaria, dato che le macchine saranno collegate tramite rete esterna.
	%Configurazione 
	\subsection{Configurazione} 
	\begin{lstlisting}[caption=File di configurazione Load Balancer NGINX] 
events {}
http {
	upstream balanceGroup1 {
		server WebServer1:80;
		server WebServer2:80;
		server WebServer3:80;
		server WebServer4:80;
	}

	server {
		listen 80;
		
		location / {
			proxy_pass http://balanceGroup1;
		}
	}
}\end{lstlisting}
	Alla riga 3 specifichiamo un gruppo di server di nome \textit{balanceGroup1}.\\
	Dalla riga 4 alla riga 7 specifichiamo gli indirizzi dei server appartenenti a \textit{balanceGroup1}, in questo caso vengono
	utilizzati degli indirizzi appartenenti alla rete \textit{Internal} ma possono essere sostituiti con normalissimi indirizzi HTTP(S).\\
	Alla riga 11 specifichiamo su che porta ascoltare.
	\paragraph{Per implementare un sistema multi host} Sostituire gli indirizzi alle righe 4-7 con gli indirizzi effettivi dei propri host su cui gira webserver (\autoref{sec:Webserver}), prestare particolare attenzione alle porte su cui rispondono i webserver (\autoref{sec:WebserverConfigurazione}).
	%Modalità di load balancing 
	\subsubsection{Modalità di load balancing} 
	NGINX supporta 3 modalità di load balancing:
	\begin{enumerate}
		\item Round Robin
		\begin{itemize}
			\item Round Robin standard (Default)
			\item Round Robin pesato
		\end{itemize}
		\item Least Connected
		\item Session Persistence (ip-hash)
	\end{enumerate}
	\paragraph{Round Robin}
	La tecnica Round Robin consiste nel dividere il carico proporzionalmente tra i vari host.
	\subparagraph{Round Robin standard (Default)} In questo caso tutti gli host hanno lo stesso peso \(\rightarrow \) il carico viene distribuito equamente tra tutti.
	\subparagraph{Round Robin pesato} In questo caso vengono specificati dei pesi per alcuni o tutti i server, il round robin distribuisce le richieste tenendo conto dei pesi specificati.\\
	In caso di omissione del peso questo viene considerato pari a 1.
	\begin{lstlisting}[caption=Round Robin pesato]
upstream balanceGroup1 {
	server WebServer1:80 weight=3;
	server WebServer2:80;
	server WebServer3:80 weight=2;
	server WebServer4:80;
}\end{lstlisting}
	In questo caso specifico i server 1 e 3 riceveranno rispettivamente il triplo e il doppio delle richieste rispetto ai server 2 e 4.
	\paragraph{Least Connected} La tecnica Least Connected tiene traccia del carico di ogni server e reindirizza al server con meno carico al momento della richiesta.\\
	Questa tecnica è molto utile nel caso in cui il tempo di risposta ad una richiesta vari significativamente in base alla richiesta, in questo caso con Least Connected
	evitiamo di caricare un server con molte richieste in elaborazione.
\begin{lstlisting}[caption=Least Connected]
upstream balanceGroup1 {
	least_conn;
	server WebServer1:80;
	server WebServer2:80;
	server WebServer3:80;
	server WebServer4:80;
}\end{lstlisting}
	\paragraph{Session Persistence (ip-hash)} Questa tecnica riassegna ad ogni ip sempre lo stesso server usando una funzione di hash per mappare ad ogni ip un server.
\begin{lstlisting}[caption=Session Persistence]
upstream balanceGroup1 {
	ip_hash;
	server WebServer1:80;
	server WebServer2:80;
	server WebServer3:80;
	server WebServer4:80;
}\end{lstlisting}
	Informazioni più dettagliate sulle tipologie di load balancing di NGINX si possono trovare nella \href{https://nginx.org/en/docs/http/load_balancing.html}{documentazione di NGINX}.
\end{document}  